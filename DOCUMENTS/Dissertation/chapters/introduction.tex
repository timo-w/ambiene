\chapter{Introduction}

% reset page numbering. Don't remove this!
\pagenumbering{arabic} 


\section{Motivations}
\todo{Add citations}

Despite being more connected to our peers than ever before, loneliness has never been so prevalent, especially in young people. Social media offers solutions to interact with one another, however, the stress of online messaging can be discouraging to many and can lead to isolation in a society where a social presence is almost necessary.

Creating music has been a proven way to bond and connect with others (CITATION), and collaborating to create sound has brought people together long before the digital age we now live in. During the COVID-19 pandemic, many turned to collaborative music-making over the internet to experience and maintain their social connections (CITATION OR EXAMPLE), so there is clearly a desire and audience for systems that provide this experience. I aim to create an online application where users can create music together and enjoy a collaborative musical bond in an online context.

This project consists of a technical goal and a psychosocial one. The first is to create a multiplayer system to procedurally generate music using user-defined parameters and modifiers, where the experience is synchronised across all connected users. The latter goal focuses on creating an engaging and relaxing social environment where users can communicate with one another nonverbally through an intuitive and immersive online collaborative music creation experience.



\section{Project Aims}

For each of these two goals, there are sub-problems which need to be solved in order to achieve the desired aims of the project.

\subsection{Technical Aims}
\begin{itemize}
    \item
        \textbf{Create a procedural music system} - Design a system which incorporates multiple instruments and tracks which when played together create a cohesive musical experience.
    \item
        \textbf{Allow sufficient user control} - The user should feel like they have control over the music experience. They should have a sufficient number of controls and modifiers to allow for a wide variety of musical contrast at the user’s discretion throughout the experience.
    \item
        \textbf{Multiplayer support} - The app should support multiple users to share the experience with. All controls and modifiers should be synchronised so each user hears the same output sound.
    \item 
        \textbf{Low-latency, high consistency} - When a user changes any control or modifier, that change should be made to all connected users with minimal latency. The rhythm/beats of the system should be consistent so that the musical immersion is not broken.
\end{itemize}

\subsection{Psychosocial Aims}
\begin{itemize}
    \item 
        \textbf{Bond between users} - Users should feel a social connection to the other connected users as they share the same musical experience.
    \item 
        \textbf{Communication through music} - Users should feel they can communicate nonverbally through the system and not rely solely on contemporary means like chat or voice communication to convey or express their intent.
    \item 
        \textbf{Relaxing environment} - The system should provide an intuitive and stress-free environment so users can feel relaxed while they create music.
    \item 
        \textbf{Accessibility first} - The system should be accessible to users with little or no music experience, any user regardless of music experience should be able to create a musical experience they can enjoy.
    \item 
        \textbf{Intuitive design} - The system should be easy to navigate and not require a user guide to use. Users should be able to learn as they create music.
\end{itemize}