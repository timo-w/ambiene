\chapter{Evaluation} 
How good is your solution? How well did you solve the general problem, and what evidence do you have to support that?

\section{Guidance}
\begin{itemize}
    \item
        Ask specific questions that address the general problem.
    \item
        Answer them with precise evidence (graphs, numbers, statistical
        analysis, qualitative analysis).
    \item
        Be fair and be scientific.
    \item
        The key thing is to show that you know how to evaluate your work, not
        that your work is the most amazing product ever.
\end{itemize}

\section{Evidence}
Make sure you present your evidence well. Use appropriate visualisations, 
reporting techniques and statistical analysis, as appropriate. The point is not
to dump all the data you have but to present an argument well supported by evidence gathered.

If you use numerical evidence, specify reasonable numbers of significant digits; don't state ``18.41141\% of users were successful'' if you only had 20 users. If you average \textit{anything}, present both a measure of central tendency (e.g. mean, median) \textit{and} a measure of spread (e.g. standard deviation, min/max, interquartile range).

You can use \texttt{siunitx} to define units, space numbers neatly, and set the precision for the whole LaTeX document. 

% setup siunitx to have two decimal places
\sisetup{
	round-mode = places,
	round-precision = 2
}

For example, these numbers will appear with two decimal places: \num{3.141592}, \num{2.71828}, and this one will appear with reasonable spacing \num{1000000}.



If you use statistical procedures, make sure you understand the process you are using,
and that you check the required assumptions hold in your case. 

If you visualise, follow the basic rules, as illustrated in Figure \ref{fig:boxplot}:
\begin{itemize}
\item Label everything correctly (axis, title, units).
\item Caption thoroughly.
\item Reference in text.
\item \textbf{Include appropriate display of uncertainty (e.g. error bars, Box plot)}
\item Minimize clutter.
\end{itemize}

See the file \texttt{guide\_to\_visualising.pdf} for further information and guidance.

\begin{figure}[htb]
    \centering
    \includegraphics[width=1.0\linewidth]{images/boxplot_finger_distance.pdf}    

    \caption{Average number of fingers detected by the touch sensor at different heights above the surface, averaged over all gestures. Dashed lines indicate
    the true number of fingers present. The Box plots include bootstrapped uncertainty notches for the median. It is clear that the device is biased toward 
    undercounting fingers, particularly at higher $z$ distances.
    }

    % use the notation fig:name to cross reference a figure
    \label{fig:boxplot} 
\end{figure}